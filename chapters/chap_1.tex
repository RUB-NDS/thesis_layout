\chapter{Introduction} \label{chap:intro}
Always start a chapter with a short but informative text about the following sections\@. Point out the relevance of the sections and create interconnections between them\@. Never ever just write a single sentence here\@. Furthermore,\ you are strongly advised to respect the hints given in this template\@.

\section{Motivation}
What is your motivation to deal with this subject\@? Which interesting problems do you expect\@? Do not abbreviate \enquote{e.\,g.\/} within a sentence,\ always write \enquote{for example}\@. However,\ within in parentheses you are allowed to abbreviate and use,\ e.\,g.,\ and, i.\,e.,\ as shown here: with a comma right before and after it\@. In addition to that,\ ensure correct spacing by using \texttt{\textbackslash,} in between\@.

\section{Related Work}
List related work \emph{and} the result of this work\@! What is the relevance of this work concerning your thesis\@? If necessary,\ \emph{emphasize} some words in your text, for example words like \emph{not} or \emph{and} are sometimes crucial for understanding\@.

\section{Contribution}
What is your contribution?

\section{Organization of this Thesis}
Please give a general overview on how your thesis is divided into sections and chapters~\dots

