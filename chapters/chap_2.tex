\chapter{Related Work}

Every thesis must contain a dedicated section on related work.
You can safely spend at least two pages for this section.

List related work \emph{and} the result of this work.
What is the relevance of this work concerning your thesis?

\textbf{Hint:} Do not manually cite author names, use \code|\citeauthor{Newsome:05:DTA}|, for example, \enquote{\citeauthor{Newsome:05:DTA}}.
It takes also care if multiple authors are used, for example, \code|\citeauthor{AviramSSHDSVAHD16}| becomes \enquote{\citeauthor{AviramSSHDSVAHD16}}.
You could also get the full list of authors by using the command \code|\citeauthor*{AviramSSHDSVAHD16}| (\enquote{\citeauthor*{AviramSSHDSVAHD16}}) but this should only be used in rare cases.

Similarly, use \code|\citeyear{Newsome:05:DTA}| to get the year (\enquote{\citeyear{Newsome:05:DTA}}).

\textbf{Tools:}
We do not make any strict guidelines regarding the tools that can be used to manage the related work.
You can freely choose your own methodology and tools.

According to our experience, the following tools can be used as a starting point to find relevant papers:
\begin{itemize}
    \item Search Engines: \href{https://scholar.google.com}{Google Scholar}, \href{https://www.webofknowledge.com/}{Web of Science}, \href{https://dl.acm.org/}{ACM Digital Library}, \href{https://ieeexplore.ieee.org/Xplore/home.jsp}{IEEExplore}, and \emph{\{Search Engine\}} + \emph{\{Conference name\} proceedings}
    \item Management Tools: \href{https://www.it-services.ruhr-uni-bochum.de/services/software/citavi.html.de}{Citavi}, \href{https://www.zotero.org/}{Zotero}, \href{https://www.mendeley.com/}{Mendeley}, and \href{https://www.jabref.org/}{JabRef}.
\end{itemize}