\chapter{Results} \label{chap:results}

It is \textbf{\gls{bcp}} to provide a brief overview on your most interesting results in the beginning.
In \autoref{tab:results} you can see the results of our \gls{sso} evaluation against \gls{xss}, \gls{sqli}, \gls{csrf} and \gls{clickjacking} attacks.

\newcommand{\vuln}{\faCircle}
\newcommand{\notvuln}{\faCircle[regular]}

\begin{table}[htbp]
    \centering
    \begin{tabular}{@{}llcccc@{}}
      \toprule
       & Browser & XSS & SQLi & CSRF & Clickjacking \\
      \midrule
      \faFirefox & Firefox & \notvuln & \vuln & \vuln & \vuln \\
      \faChrome  & Chrome & \vuln & \notvuln & \vuln & \vuln \\
      \faSafari  & Safari & \vuln & \vuln & \notvuln & \vuln \\
      \bottomrule
    \end{tabular}
    \label{tab:results}
    \caption{Write a short message that everyone should understand, e.g., \enquote{All browsers were found vulnerable (\vuln{}) to Clickjacking} instead of simply writing \enquote{Evaluation results.}. Also describe what the symbols mean.}
  \end{table}