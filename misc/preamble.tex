\setcounter{tocdepth}{3}  % Inhaltsverzeichnis bis Subsubsection
\setcounter{secnumdepth}{3} % Nummerierung bis Subsubsection

% General stuff
\usepackage[utf8]{inputenc} % CHANGE HERE IF NECESSARY
\usepackage[T1]{fontenc}
\usepackage[ngerman, english]{babel} % last language given is used (here: english)
\usepackage{lmodern}
%\usepackage{microtype}
\usepackage{ifpdf}
\usepackage{ifthen}

% Set date here 
%\day=6 \month=6 \year=2012

% Set name and title
\author{Insert your name here}
\title{Insert title here}
\date{\today}

%%%%%% %%%%%%

% Load packages ...

% Corporate Design
\usepackage{eso-pic}
\usepackage{color}
% Comment out if the RUB fonts are installed
% Link: https://noc.rub.de/~jobsanzl/latex/rubtexfonts-0.4.tar.gz
%\usepackage{rubfonts2009} 
\newcommand{\setrubfontnormal}[1]{\fontfamily{rubscala}\fontsize{#1}{1}\selectfont}
\newcommand{\setrubfontextra}[1]{\fontfamily{rubflama}\fontsize{#1}{1}\selectfont}
\definecolor{rubgreen}{cmyk}{0.5,0,1,0}
\definecolor{rubblue}{cmyk}{1,0.5,0,0.6}

% Figures
\usepackage{graphicx}
\usepackage{subfig}
\usepackage{placeins}

% Tables
\usepackage{booktabs}
\usepackage{marvosym}
\usepackage{multirow}

% Math stuff and units
\usepackage{latexsym,amsmath, amssymb, amsfonts, upgreek}
\usepackage{siunitx}
\newcommand{\mathup}{\mathrm}

% Glossary
\usepackage[nonumberlist, acronym, toc]{glossaries}

% Enable quotes by \enquote{}
\usepackage[babel,english=american, german=quotes]{csquotes}

% Necessary for frontpage, allows to create automata and fancy graphics
\usepackage{tikz}

% Protocols and bytefields
\usepackage{protocol}
\usepackage{bytefield}

% Source code listings
\newcommand{\code}[1]{\texttt{#1}}
\definecolor{colIdentifier}{rgb}{0,0,0}
\definecolor{colComments}{rgb}{0.5,0.5,0.5}
\definecolor{colKeys}{rgb}{0,0,1}
\definecolor{colString}{rgb}{0,0.6,0}

\usepackage{caption}
\usepackage{listings}
\lstset{%
	float=hbp,%
	basicstyle=\ttfamily\scriptsize, %
	identifierstyle=\color{colIdentifier}, %
	keywordstyle=\color{colKeys}, %
	stringstyle=\color{colString}, %
	commentstyle=\color{colComments}, %
	columns=flexible, %
	tabsize=2, %
	aboveskip={1.5\baselineskip}, %
	frame=single, %
	extendedchars=true, %
	showspaces=false, %
	showstringspaces=false, %
	numberstyle=\tiny, %
	breaklines=true, %
	backgroundcolor=, %
	breakautoindent=true, %
	captionpos=b%
}

% Algorithms
\usepackage[ruled, vlined, linesnumbered,algochapter,algo2e]{algorithm2e}

% Format page foot and header
\usepackage{scrlayer-scrpage}
\clearscrheadings
\clearscrheadfoot
\automark[section]{chapter}
\ohead{\pagemark}
\ihead{\headmark}
\pagestyle{scrheadings}

%% use some standards for mathematical expressions:
\newcommand{\red}{{\rm red}}
\newtheorem{theorem}{Theorem}[section]
\newtheorem{lemma}[theorem]{Lemma}
\newtheorem{proposition}[theorem]{Proposition}
\newtheorem{corollary}[theorem]{Corollary}
% \newtheorem{definition}[theorem]{Definition}
\newtheorem{algorithm}[theorem]{Algorithm}
\newenvironment{example}{\begin{quote}{\bf Example:}}{\end{quote}}

% BIBTEX, http://mirrors.ctan.org/biblio/bibtex/contrib/babelbib/babelbib.pdf
\usepackage{babelbib}
\usepackage{url}
\def\UrlBreaks{\do\/\do-}

% \setbibpreamble{{\large Seitenzahlen, auf denen ein Eintrag referenziert wird, werden am Ende eines jeden Eintrags angegeben.}\newline} % Wegen der pagebackref-Option des hyperref-Packets, wird vielen nicht direkt klar was das soll http://projekte.dante.de/DanteFAQ/Verzeichnisse#16

% gray definition boxes, that whay you'll find them in the text
\usepackage{shadethm}
\newshadetheorem{sthm}[figure]{Definition}
\newenvironment{definition}[1][]{
   \definecolor{shadethmcolor}{rgb}{.9,.9,.9}
   \begin{sthm}[#1]
 }{\end{sthm}}

% experimental
%\usepackage{scrhack}

% Hyperlinks and menu for your document
\usepackage[breaklinks,hyperindex,colorlinks,anchorcolor=black,citecolor=black,filecolor=black,linkcolor=black,menucolor=black,urlcolor=black,pdftex]{hyperref} % pagebackref: Add page number to the references where they can be found
% DO NOT LOAD ANY OF YOUR PACKAGES BEYOND THIS PACKAGE

\makeatletter
\AtBeginDocument{
 \hypersetup{
   pdftitle = {\@title},
   pdfauthor = {\@author},
   pdfsubject={\@title},
   pdfkeywords={SAML, add more}, % CHANGE HERE
%    unicode={true},
 }
}
\makeatother

% Use the same counter for tables and figures
\makeatletter
\AtBeginDocument{
\let\c@table\c@figure
\let\c@lstlisting\c@table
\let\c@algocf\c@lstlisting
}
\makeatother

\ifpdf
	\hypersetup{linktocpage=false} 	% false=links are section names, true=links are page numbers, IMPORTANT: in dvi2ps mode, 'true' is required!
\else
	\hypersetup{linktocpage=true} 		% false=links are section names, true=links are page numbers, IMPORTANT: in dvi2ps mode, 'true' is required!
	\usepackage[hyphenbreaks]{breakurl}
\fi
